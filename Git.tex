
Git 초기 설정
  - git config --global user.name "이름"
  - git config --global user.email "GitHub 등록 이메일"
  - git remote add origin "주소(GitHub 등록 경로)"
    * 경로는 HTTPS와 SSH 등으로 설정 가능함

Git 환경 구축(Local)
  - git init 
    * .git 폴더를 생성하고 해당 폴더를 관리할 수 있도록 초기화 한다.
  - git add .
  - git commit -m "수정내용"

GitHub 업로드
  - git push origin main

GitHub 다운로드
  - git pull origin main

Git 상태 확인 명령어
  - git status
    * 스테이지에 등록/미등록 현황 및 GitHub 업데이트 상태를 볼 수 있다.
  - git remote -v 
    * GitHub와 연결 정보를 볼 수 있다.
  - git log
    * 현재 브랜치에서 모든 버전 이력을 보여준다.
  - git log filename
    * 지정된 파일의 버전 이력을 보여준다.

Branch 관리
  - git branch branch_name
    * 새 브랜치를 생성합니다:
  - git checkout branch_name
    * 현재 브랜치를 닫고 다른 브랜치로 변경한다.
  - git checkout -b branch_name
    * 새로운 브랜치를 생성하고 자동으로 전환해 준다.
  - git branch
    * 모든 브랜치의 목록을 보여준다.

에러 조치

- 에러내용 
   fatal: unable to access 'https://github.com/HyunyunLee/MyDocs.git/': SSL certificate problem: self signed certificate in certificate chain
- 조치방법
   SSL 보안서버 인증서를 구매하여 사용하지 않고, Open SSL 인증서를 사용한 경우 git push시 SSL에러가 발생한다.
   이를 해결하기 위해 CA에서 인증하는 절차를 무시하는 방법이 있다.
   git config --global http.sslVerify false
