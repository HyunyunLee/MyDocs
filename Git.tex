새로운 깃 저장소 초기 설정: Git init
모든 코드 내용은 저장소 안에 추적됩니다. 깃 저장소를 초기 설정하려면 해당 프로젝트 폴더 안에 이 명령어를 사용하세요. 이것이 .git 폴더를 만들어 줄 것입니다.

git init

Git add
이 명령어는 하나 또는 모든 변경 파일들을 본 무대 영역으로 더합니다.

어떤 하나의 특정 파일을 올리기 위해서는:

git add filename

신규 또는 수정되거나 삭제된 파일들을 올리기 위해서는:

git add -A

신규 또는 수정 파일들을 올리기 위해서는:

git add .

수정 또는 삭제된 파일들을 올리기 위해서는:

git add -u

Git commit
이 명령어는 버전 이력을 파일 안에 기록합니다. -m이 뜻하는 것은 어떤 커밋 메세지가 뒤따른다는 의미입니다. 이 메세지는 커스텀이며 반드시 이것을 동료에게 알리는 용도로 또는 미래의 스스로에게 무엇이 해당 커밋 안에 더해졌는지 알리기 위해 사용해야 합니다.

git commit -m "your text"

Git status
이 명령어는 파일들을 초록색과 빨간색으로 리스트해 줄 것입니다. 초록색 파일들은 무대로 올려졌지만 아직 커밋되지 않은 것들입니다. 빨간색으로 표시된 파일들은 무대로 아직 올려지지 않은 것들입니다.

git status

Git branch branch_name
이것은 새 브랜치를 생성합니다:

git branch branch_name

Git checkout branch_name
어떤 브랜치에서 다른 브랜치로 변경하려면:

git checkout branch_name

Git checkout -b branch_name
새로운 브랜치를 생성하고 그것으로 자동 전환하려면:

git checkout -b branch_name

이것은 간략하게:
git branch branch_name
git checkout branch_name


Git branch
모든 브랜치를 리스트하고 현재 어떤 브랜치에 있는지 확인하려면:

git branch

Git log
이 명령어는 현재 브랜치에서 모든 버전 이력을 리스트해줄 것입니다:

git log

Push와 Pull
Git push
이 명령어는 커밋된 변경점들을 원격 저장소에 보냅니다:

git push

Git pull
원격 저장소에서 개인 컴퓨터로 변경점들을 가져오려면:

git pull





에러 조치

fatal: unable to access 'https://github.com/HyunyunLee/MyDocs.git/': SSL certificate problem: self signed certificate in certificate chain

SSL 보안서버 인증서를 구매하여 사용하지 않고, Open SSL 인증서를 사용한 경우 git push시 SSL에러가 발생한다.
이를 해결하기 위해 CA에서 인증하는 절차를 무시하는 방법이 있다.

git config --global http.sslVerify false