
Git 초기 설정
  - git config --global user.name "이름"
  - git config --global user.email "GitHub 등록 이메일"
  - git remote add origin "주소(GitHub 등록 경로)"
    * 경로는 HTTPS와 SSH 등으로 설정 가능함
    [remote관리]
    - git remote rm origin
      * 등록된 origin을 삭제한다.
    - git remote -v 
      * GitHub와 연결 정보를 볼 수 있다.
    - git ls-remote
      * GitHub와 연결 정보를 볼 수 있다.

Git 환경 구축(Local)
  - git init 
    * .git 폴더를 생성하고 해당 폴더를 관리할 수 있도록 초기화 한다.
  - git add .
  - git commit -m "수정내용"

GitHub 업로드
  - git push origin main
    * upload 및 merge를 동시에 수행해준다.

GitHub 다운로드
  - git pull origin main
    * fetch 및 merge를 동시에 수행해준다.

Git 상태 확인 명령어
  - git status
    * 스테이지에 등록/미등록 현황 및 GitHub 업데이트 상태를 볼 수 있다.
  - git log
    * 현재 브랜치에서 모든 버전 이력을 보여준다.
  - git log filename
    * 지정된 파일의 버전 이력을 보여준다.
  - git commit --amend -m "커밋 메시지"
    * 최종 커밋 로그 변경

Reset(해쉬값은 git log 명령을 통해 확인한다.)
  - Soft: 헤더만 제거한다.(commit 전 상태로 돌린다.)
    -> 커밋 로그 변경시 사용
    git reset --soft (이전 버전의 해쉬값(앞에 5자리 정도 입력))
  - Mixed: 스테이지, 헤더를 제거한다.(add 전 상태로 되돌린다.)   <== 활용 빈도 낮음
    -> 작업 영역의 파일 내용을 변경할때 사용(파일 내용 수정하고 싶을때)
    git reset --mixed (이전 버전의 해쉬값(앞에 5자리 정도 입력))
  - Hard: 파일, 스테이지, 헤더 모두 제거한다.(초기 상태로 되돌린다.)
    -> 이전 버전으로 돌아갈때 사용
    git reset --hard (이전 버전의 해쉬값(앞에 5자리 정도 입력))
    참고) 작업 영역의 파일 까지도 삭제된다.

Reflog
  - git reflog
    -> 모든 커밋 로그를 보여준다.
    -> 목록에서 되돌리고 싶은 버전의 해쉬값을 찾아 아래와 같이 입력하면 복구된다.
       git reset --hard 해쉬값

Branch 관리
fast-forward merge : main노드에 생성된 노드가 없을 경우 병합하는 방법으로 branch의 마지막 노드를 main 노드에 병합한다.
3way merge : 분기점 기준으로 기준 노드와 main의 첫번째 노드, branch의 첫번째 노드를 병합한다.
  - git branch branch_name
    * 새 브랜치를 생성합니다:
  - git checkout branch_name
    * 현재 브랜치를 닫고 다른 브랜치로 변경한다.
  - git checkout -b branch_name
    * 새로운 브랜치를 생성하고 자동으로 전환해 준다.
  - git branch
    * 모든 브랜치의 목록을 보여준다.

Merge
  - git merge sub-branch
    -> sub-branch 내용을 main에 병합한다.

Rebase : 코드를 재정리 하는것
  - git rebase -i HEAD~3

Clone
  - git clone "주소(GitHub 등록 경로)"
    * 빈 디렉토리에서 실행하면 git init, git remote ~, pit pull을 한 번에 수행해준다.
    * fetch origin도 수행해준다.
    * 초기 세팅시 편리함

branch를 만들었을때 다운로드 방법
  <방법1>
  - git checkout -b (branch 이름)
    * 브랜치를 만들고 브렌치로 이동한다.
  - git fetch origin
    * origin에 있는 모든 내용을 받는다.(머지는 하지 않음)
  - git merge origin/(branch 이름)
    * 원하는 branch를 병합한다.
  <방법2>
  - git checkout -b (branch 이름)
    * 브랜치를 만들고 브렌치로 이동한다.(브랜치 생성)
  - git pull origin (branch 이름)(브렌치 다운로드 및 머지)
    * 브랜치를 다운로드한다.
  <방법3> <== 자주 사용함
  - git fetch origin
    * branch를 다운받는다.(모든 브랜치 다운로드)
  - git checkout -b (branch 이름) origin/(branch 이름)
    * 브렌치를 만들고 병합까지 수행해준다.(브랜치 생성 및 머지)

branch 작업 완료 후 병합하기(rebase 명령을 사용해도 됨)
  - git merge --squash (branch 이름)
    * main 브렌치에서 실행, commit이 안된 상태로 병합됨
  - git commit -m "내용"
  - git push origin main

에러 조치

- 에러내용 
   fatal: unable to access 'https://github.com/HyunyunLee/MyDocs.git/': SSL certificate problem: self signed certificate in certificate chain
- 조치방법
   SSL 보안서버 인증서를 구매하여 사용하지 않고, Open SSL 인증서를 사용한 경우 git push시 SSL에러가 발생한다.
   이를 해결하기 위해 CA에서 인증하는 절차를 무시하는 방법이 있다.
   git config --global http.sslVerify false
