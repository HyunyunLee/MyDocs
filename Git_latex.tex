\documentclass[12pt,a4paper]{oblivoir}

\usepackage{fapapersize}
\usefastocksize{210mm,297mm} % 공급 용지 크기
\usefapapersize{*,*,25mm,*,23mm,28mm} % 용지싸이즈 2개, 좌, 우, 상, 하

\title{GitHub 사용방법}
\author{이현연}
\date{2023년}

% \SetHangulspace{1.3}{1.1}  % 행간 조절(이해 못했음)
% \ResetHangulspace{1.3}{1.1}

% \makepagestyle{myheading}
% \makeevenfoot{myheading}{\thepage / \thetotalpages}{}{}
% \makeoddfoot{myheading}{}{}{\thepage}
% \makeevenhead{myheading}{2023-1\quad 전자회로해석 II}{}{95112222 홍길동}
% \makeoddhead{myheading}{2000000 홍길동}{}{1차 스터디그룹 발표}
% \pagestyle{myheading}

\begin{document}
\maketitle

\section[Git 초기 설정]{Git 초기 설정}
\begin{itemize}
\item git config --global user.name ``이름''
\item git config --global user.email ``GitHub 등록 이메일''
\item git remote add origin ``주소(GitHub 등록 경로)'' \\  ☞ 경로는 HTTPS와 SSH 등으로 설정 가능함
\end{itemize}


[ remote관리 ]
\begin{itemize}
\item git remote rm origin \\  ☞ 등록된 origin을 삭제한다.
\item git remote -v \\  ☞ GitHub와 연결 정보를 볼 수 있다.
\item git ls-remote \\  ☞ GitHub와 연결 정보를 볼 수 있다.
\end{itemize}
 
\section[Git 환경 구축(local)]{Git 환경 구축(local)}
\begin{itemize}
\item git init  \\  ☞ .git 폴더를 생성하고 해당 폴더를 관리할 수 있도록 초기화 한다.
\item git add .
\item git commit -m ``수정내용''
\end{itemize}

\section[GitHub 업로드]{GitHub 업로드}
\begin{itemize}
\item git push origin main \\  ☞ upload 및 merge를 동시에 수행해준다.
\item git push --all \\  ☞ 모든 branch를 모두 업로드 해준다.
\item git push -f origin branch\_name \\  ☞ 해당 브랜치를 강제로 푸쉬한다.
\end{itemize}

\section[GitHub 다운로드]{GitHub 다운로드}
\begin{itemize}
\item git pull origin main \\  ☞ fetch 및 merge를 동시에 수행해준다.
\end{itemize}

\section[Git 상태 확인 명령어]{Git 상태 확인 명령어}
\begin{itemize}
\item git status \\  ☞ 스테이지에 등록/미등록 현황 및 GitHub 업데이트 상태를 볼 수 있다.
\item git log \\  ☞ 현재 브랜치에서 모든 버전 이력을 보여준다.
\item git log filename \\  ☞ 지정된 파일의 버전 이력을 보여준다.
\item git commit --amend -m ``커밋 메시지'' \\  ☞ 최종 커밋 로그 변경
\end{itemize}

\section[Git Reset]{Git Reset}
\subsection{Soft : 헤더만 제거한다.(commit 전 상태로 돌린다.)}  
\begin{itemize}
\item git reset --soft (이전 버전의 해쉬값(앞에 5자리 정도 입력)) \\  ☞ 커밋 로그 변경시 사용
\end{itemize}

\subsection{Mixed : 스테이지, 헤더를 제거한다.(add 전 상태로 되돌린다.)   ← 활용 빈도 낮음}
\begin{itemize}
\item git reset --mixed (이전 버전의 해쉬값(앞에 5자리 정도 입력)) \\  ☞ 작업 영역의 파일 내용을 변경할때 사용(파일 내용 수정하고 싶을때)
\end{itemize}
\subsection{Hard : 파일, 스테이지, 헤더 모두 제거한다.(초기 상태로 되돌린다.)}
\begin{itemize}
\item git reset --hard (이전 버전의 해쉬값(앞에 5자리 정도 입력)) \\  ☞ 작업 영역의 파일 까지도 삭제된다.(이전 버전으로 돌아갈때 사용)
\end{itemize}

\section[Git Reflog]{Git Reflog}
\begin{itemize}
\item - git reflog \\ ☞ 모든 커밋 로그를 보여준다. \\  ☞ 목록에서 되돌리고 싶은 버전의 해쉬값을 찾아 아래와 같이 입력하면 복구된다. \\  ☞ git reset --hard 해쉬값
\end{itemize}

\section[Branch 관리]{Branch 관리}
\hbox{
fast-forward merge : main노드에 생성된 노드가 없을 경우 병합하는 방법으로 branch의 마지막 노드를 main 노드에 병합한다.\\ 
3way merge : 분기점 기준으로 기준 노드와 main의 첫번째 노드, branch의 첫번째 노드를 병합한다.\\
}

\begin{itemize}
\item git branch branch\_name \\  ☞ 새 브랜치를 생성합니다.
\item git checkout branch\_name \\  ☞ 현재 브랜치를 닫고 다른 브랜치로 변경한다.
\item git checkout -b branch\_name \\  ☞ 새로운 브랜치를 생성하고 자동으로 전환해 준다.
\item git branch \\  ☞ 모든 브랜치의 목록을 보여준다.
\end{itemize}

\section[Merge]{Merge}
\begin{itemize}
\item git merge sub-branch \\  ☞ sub-branch 내용을 main에 병합한다. \\  ☞ --no-ff 옵션 : fast-forward Merge 경우 커밋을 남기도록 설정한다.
\end{itemize}

\section[Rebase]{Rebase}
\begin{itemize}
\item git rebase -i HEAD-3 \\  ☞ Rebase : 코드를 재정리 하는것
\end{itemize}

\section[Clone]{Clone}
\begin{itemize}
\item git clone ``주소(GitHub 등록 경로)'' \\  ☞ 빈 디렉토리에서 실행하면 git init, git remote -, pit pull을 한 번에 수행해준다. \\  ☞ fetch origin도 수행해준다. \\  ☞ 초기 세팅시 편리함
\end{itemize}

\section[branch를 만들었을때 다운로드 방법]{branch를 만들었을때 다운로드 방법}
\subsection{방법1}
\begin{itemize}
\item git checkout -b (branch 이름) \\  ☞ 브랜치를 만들고 브렌치로 이동한다.
\item git fetch origin \\  ☞ origin에 있는 모든 내용을 받는다.(머지는 하지 않음)
\item git merge origin/(branch 이름) \\  ☞ 원하는 branch를 병합한다.
\end{itemize}
\subsection{방법2}
\begin{itemize}
  \item git checkout -b (branch 이름) \\  ☞ 브랜치를 만들고 브렌치로 이동한다.(브랜치 생성)
\item git pull origin (branch 이름)(브렌치 다운로드 및 머지) \\  ☞ 브랜치를 다운로드한다.
\end{itemize}
\subsection{방법3 - 자주 사용됨}
\begin{itemize}
\item git fetch origin \\  ☞ branch를 다운받는다.(모든 브랜치 다운로드)
\item git checkout -b (branch 이름) origin/(branch 이름) \\  ☞ 브렌치를 만들고 병합까지 수행해준다.(브랜치 생성 및 머지)
\end{itemize}
\section[branch 작업 완료 후 병합하기]{branch 작업 완료 후 병합하기}
branch 작업 완료 후 병합하기(rebase 명령을 사용해도 됨)
\begin{itemize}
\item git merge --squash (branch 이름) \\  ☞ main 브렌치에서 실행, commit이 안된 상태로 병합됨
\item git commit -m ``내용''
\item git push origin main
\end{itemize}

\section[branch 머지 완료 후 제거(협업시)]{branch 머지 완료 후 제거(협업시)}
\begin{itemize}
\item git branch --delete origin branch\_name \\  ☞ 관리자에게 보여지는 작업용 브랜치를 삭제할때 사용한다.(팀원 git에는 남아 있음)
\end{itemize}

\section[태그 달기]{태그 달기}
\begin{itemize}
\item git tag (프로그램명+버전)☞ \\  ☞ 버전 옆에 태그를 붙여준다.
\item git push --tags origin main  \\  ☞ 만들어진 태그까지 업로드한다.
\end{itemize}

\section[소규모 협업]{소규모 협업}
\subsection{팀장}
\begin{enumerate}
\item main 브랜치 초기 설정
\item dev 브랜치 생성
\item main, dev 브랜치 보호
\item 프로젝트 참여 인원 등록
\item 팀원 작업 완료후 pr 접수시 검토 및 marge
\item 최종 작업 완료 후 main 브랜치로 병합
\end{enumerate}

\subsection{팀원}
\begin{enumerate}
\item topic 브랜치 생성 
\begin{itemize}
\item   git checkout -b login\_topic
\end{itemize}
\item 개발 진행 및 완료 \\ <코드 수정 진행>
\begin{itemize}
\item git add .
\item git commit -m ``로그''
\end{itemize}
\item 로그 정리(rebase)
\begin{itemize}
\item   git rebae -i HEAD-2
\end{itemize}
\item topic 브랜치 push
\begin{itemize}
\item git push orgin login\_topic \\  ☞ 상황에 따라 -f(강제올리기) 옵션 붙일것
\end{itemize}
\item pr 요청
\begin{itemize}
\item GitHub에서 수행
\end{itemize}
\item pr 승인 및 marge(팀장)
\begin{itemize}
\item GitHub에서 수행
\end{itemize}
\item marge 완료 후 원격 branch 삭제
\begin{itemize}
\item git push --delete origin login\_topic
\end{itemize}
\item 작업 완료된 dev branch 동기화(pull)
\begin{itemize}
\item git pull origin dev
\end{itemize}
\end{enumerate}

\section[에러 조치]{에러 조치}
\begin{itemize}
  \item git push --delete origin login\_topic
\end{itemize}
  
\begin{itemize}
\item 에러내용 : fatal: unable to access `https://github.com/HyunyunLee/MyDocs.git/': SSL certificate problem: self signed certificate in certificate chain
\item 조치방법 \\
SSL 보안서버 인증서를 구매하여 사용하지 않고, Open SSL 인증서를 사용한 경우 git push시 SSL에러가 발생한다.
이를 해결하기 위해 CA에서 인증하는 절차를 무시하는 방법이 있다.
git config --global http.sslVerify false
\end{itemize}
\end{document}