\documentclass[12pt,a4paper,twoside, footnote]{oblivoir}

\usepackage{fapapersize}
\usefastocksize{210mm,297mm} % 공급 용지 크기
\usefapapersize{*,*,25mm,*,23mm,28mm} % 용지싸이즈 2개, 좌, 우, 상, 하

\title{GitHub 사용방법}
\author{이현연}
\date{2023년}

% \SetHangulspace{1.3}{1.1}  % 행간 조절(이해 못했음)
% \ResetHangulspace{1.3}{1.1}

% \makepagestyle{myheading}
% \makeevenfoot{myheading}{\thepage / \thetotalpages}{}{}
% \makeoddfoot{myheading}{}{}{\thepage}
% \makeevenhead{myheading}{2023-1\quad 전자회로해석 II}{}{95112222 홍길동}
% \makeoddhead{myheading}{2000000 홍길동}{}{1차 스터디그룹 발표}
% \pagestyle{myheading}

\begin{document}
\maketitle

\section[Git 초기 설정]{Git 초기 설정}
- git config --global user.name "이름"

- git config --global user.email "GitHub 등록 이메일"

- git remote add origin "주소(GitHub 등록 경로)"

* 경로는 HTTPS와 SSH 등으로 설정 가능함

[ remote관리 ]

- git remote rm origin

 * 등록된 origin을 삭제한다.
- git remote -v 
 * GitHub와 연결 정보를 볼 수 있다.
    
- git ls-remote
 * GitHub와 연결 정보를 볼 수 있다.
 
\section[Git 환경 구축(local)]{Git 환경 구축(local)}
  - git init 
    * .git 폴더를 생성하고 해당 폴더를 관리할 수 있도록 초기화 한다.
  - git add .
  - git commit -m "수정내용"

\section[GitHub 업로드]{GitHub 업로드}
  - git push origin main
    * upload 및 merge를 동시에 수행해준다.
  - git push --all
    * 모든 branch를 모두 업로드 해준다.
  - git push -f origin branch_name
    * 해당 브랜치를 강제로 푸쉬한다.

\section[GitHub 다운로드]{GitHub 다운로드}
  - git pull origin main
    * fetch 및 merge를 동시에 수행해준다.

\section[Git 상태 확인 명령어]{Git 상태 확인 명령어}
  - git status
    * 스테이지에 등록/미등록 현황 및 GitHub 업데이트 상태를 볼 수 있다.
  - git log
    * 현재 브랜치에서 모든 버전 이력을 보여준다.
  - git log filename
    * 지정된 파일의 버전 이력을 보여준다.
  - git commit --amend -m "커밋 메시지"
    * 최종 커밋 로그 변경

\section[Git Reset]{Git Reset}
  (해쉬값은 git log 명령을 통해 확인한다.)
  - Soft: 헤더만 제거한다.(commit 전 상태로 돌린다.)
    -> 커밋 로그 변경시 사용
    git reset --soft (이전 버전의 해쉬값(앞에 5자리 정도 입력))
  - Mixed: 스테이지, 헤더를 제거한다.(add 전 상태로 되돌린다.)   <== 활용 빈도 낮음
    -> 작업 영역의 파일 내용을 변경할때 사용(파일 내용 수정하고 싶을때)
    git reset --mixed (이전 버전의 해쉬값(앞에 5자리 정도 입력))
  - Hard: 파일, 스테이지, 헤더 모두 제거한다.(초기 상태로 되돌린다.)
    -> 이전 버전으로 돌아갈때 사용
    git reset --hard (이전 버전의 해쉬값(앞에 5자리 정도 입력))
    참고, 작업 영역의 파일 까지도 삭제된다.

\section[Git Reflog]{Git Reflog}
  - git reflog
    -> 모든 커밋 로그를 보여준다.
    -> 목록에서 되돌리고 싶은 버전의 해쉬값을 찾아 아래와 같이 입력하면 복구된다.
       git reset --hard 해쉬값

\section[Branch 관리]{Branch 관리}
fast-forward merge : main노드에 생성된 노드가 없을 경우 병합하는 방법으로 branch의 마지막 노드를 main 노드에 병합한다.
3way merge : 분기점 기준으로 기준 노드와 main의 첫번째 노드, branch의 첫번째 노드를 병합한다.
  - git branch branch_name
    * 새 브랜치를 생성합니다:
  - git checkout branch_name
    * 현재 브랜치를 닫고 다른 브랜치로 변경한다.
  - git checkout -b branch_name
    * 새로운 브랜치를 생성하고 자동으로 전환해 준다.
  - git branch
    * 모든 브랜치의 목록을 보여준다.

\section[Merge]{Merge}
  - git merge sub-branch
    -> sub-branch 내용을 main에 병합한다.
    * --no-ff 옵션 : fast-forward Merge 경우 커밋을 남기도록 설정한다.

\section[Rebase]{Rebase}
Rebase : 코드를 재정리 하는것
  - git rebase -i HEAD~3

\section[Clone]{Clone}
  - git clone "주소(GitHub 등록 경로)"
    * 빈 디렉토리에서 실행하면 git init, git remote ~, pit pull을 한 번에 수행해준다.
    * fetch origin도 수행해준다.
    * 초기 세팅시 편리함

\section[branch를 만들었을때 다운로드 방법]{branch를 만들었을때 다운로드 방법}
\subsection{방법1}
  - git checkout -b (branch 이름)
    * 브랜치를 만들고 브렌치로 이동한다.
  - git fetch origin
    * origin에 있는 모든 내용을 받는다.(머지는 하지 않음)
  - git merge origin/(branch 이름)
    * 원하는 branch를 병합한다.
\subsection{방법2}
  - git checkout -b (branch 이름)
    * 브랜치를 만들고 브렌치로 이동한다.(브랜치 생성)
  - git pull origin (branch 이름)(브렌치 다운로드 및 머지)
    * 브랜치를 다운로드한다.
\subsection{방법3}
  자주 사용함
  - git fetch origin
    * branch를 다운받는다.(모든 브랜치 다운로드)
  - git checkout -b (branch 이름) origin/(branch 이름)
    * 브렌치를 만들고 병합까지 수행해준다.(브랜치 생성 및 머지)

\section[branch 작업 완료 후 병합하기]{branch 작업 완료 후 병합하기}
  branch 작업 완료 후 병합하기(rebase 명령을 사용해도 됨)
  - git merge --squash (branch 이름)
    * main 브렌치에서 실행, commit이 안된 상태로 병합됨
  - git commit -m "내용"
  - git push origin main

\section[branch 머지 완료 후 제거(협업시)]{branch 머지 완료 후 제거(협업시)}
  - git branch --delete origin branch_name
    * 관리자에게 보여지는 작업용 브랜치를 삭제할때 사용한다.(팀원 git에는 남아 있음)

\section[태그 달기]{태그 달기}
  - git tag (프로그램명+버전)*
    * 버전 옆에 태그를 붙여준다.
  - git push --tags origin main 
    * 만들어진 태그까지 업로드한다.


\section[소규모 협업]{소규모 협업}
\subsection{팀장}
1. main 브랜치 초기 설정
2. dev 브랜치 생성
3. main, dev 브랜치 보호
4. 프로젝트 참여 인원 등록
5. 팀원 작업 완료후 pr 접수시 검토 및 marge
6. 최종 작업 완료 후 main 브랜치로 병합
\subsection{팀원}
1. topic 브랜치 생성
    - git checkout -b login_topic
2. 개발 진행 및 완료
    <코드 수정 진행>
    - git add .
    - git commit -m "로그"
3. 로그 정리(rebase)
    - git rebae -i HEAD~2
4. topic 브랜치 push
    - git push orgin login_topic
      * 상황에 따라 -f(강제올리기) 옵션 붙일것
5. pr 요청
    - GitHub에서 수행
6. pr 승인 및 marge(팀장)
    - GitHub에서 수행
7. marge 완료 후 원격 branch 삭제
    - git push --delete origin login_topic
8. 작업 완료된 dev branch 동기화(pull)
    - git pull origin dev

\section[에러 조치]{에러 조치}
- 에러내용 
   fatal: unable to access 'https://github.com/HyunyunLee/MyDocs.git/': SSL certificate problem: self signed certificate in certificate chain
- 조치방법
   SSL 보안서버 인증서를 구매하여 사용하지 않고, Open SSL 인증서를 사용한 경우 git push시 SSL에러가 발생한다.
   이를 해결하기 위해 CA에서 인증하는 절차를 무시하는 방법이 있다.
   git config --global http.sslVerify false
\end{Large}
\end{document}