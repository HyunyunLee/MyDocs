%\documentclass{article}
\documentclass[12pt,a4paper,twoside, footnote]{oblivoir}
%\usepackage{graphicx}
%\usepackage{kotex}
\usepackage{fapapersize}
\usefastocksize{210mm,297mm} % 공급 용지 크기
\usefapapersize{*,*,25mm,*,23mm,28mm} % 용지싸이즈 2개, 좌, 우, 상, 하

\usepackage{amsthm}
\theoremstyle{plain}
\newtheorem{theorem}{Theorem}[section]

\theoremstyle{definition}
\newtheorem{definition}[theorem]{Definition}

\theoremstyle{remark}
\newtheorem{remark}[theorem]{Remark}


\title{LaTex 사용방법}
\author{홍길동}
\date{2023년}

\SetHangulspace{1.3}{1.1}  % 행간 조절(이해 못했음)
\ResetHangulspace{1.3}{1.1}

\makepagestyle{myheading}
\makeevenfoot{myheading}{\thepage / \thetotalpages}{}{}
\makeoddfoot{myheading}{}{}{\thepage}
\makeevenhead{myheading}{2023-1\quad 전자회로해석 II}{}{95112222 홍길동}
\makeoddhead{myheading}{2000000 홍길동}{}{1차 스터디그룹 발표}
\pagestyle{myheading}

\begin{document}
\maketitle
Hello World!!
\begin{center}
\LARGE LARGE test
$\sqrt{2}$
$\sqrt[3]{2}$

안녕하세요

`apple'
``banana''

\begin{theorem}
    AA
\end{theorem}

\begin{definition}
    AA
\end{definition}

\section[목차에 표시되는 내용입니다.]{첫 번째 절}
첫번째 절입니다.
이것은 시험입니다.

\section*{두 번째 절}
두번째 절입니다.
이것은 시험입니다.

\section{세 번째 절}
\subsection{세번째 절입니다.}
이것은 시험입니다.

\section{네 번째 절}
\subsection{글자크기}
{\tiny triy : 10포인트} 기준 6포인트

\scriptsize scriptsize : 10포인트 기준 7포인트

\footnotesize footnotesie : 10포인트 기준 8포인트

\small small : 10포인트 기준 9포인트

\normalsize normalsize : 10포인트 기준 10포인트

\large large : 10포인트 기준 10.95포인트

\Large Large : 10포인트 기준 12포인트

\LARGE LARGE : 10포인트 기준 14.4포인트

\huge huge : 10포인트 기준 17.28포인트

\Huge Huge : 10포인트 기준 20.74포인트

\begin{Large}
이곳은 지역적으로
글자 크기를 설정했습니다.

\begin{flushleft}
\textrm{안녕하세요 Hello World!!}

\textsf{안녕하세요\\ Hello World!!}

\texttt{안녕하세요 Hello World!!}
\end{flushleft}

\begin{flushright}

\textmd{안녕하세요 Hello World!!}

\textbf{안녕하세요\\ Hello World!!}

\textup{안녕하세요 Hello World!!}

\end{flushright}

\begin{center}

\textit{안녕하세요\footnote{한국식 인사 표현입니다.} Hello World!!}

\textsl{안녕하세요\\ Hello World!!}

\textsc{안녕하세요 Hello World!!}

\emph{안녕하세요 Hello World!!}

\textnormal{안녕하세요 Hello World!!}
\end{center}
\end{Large}
\end{center}
\end{document}